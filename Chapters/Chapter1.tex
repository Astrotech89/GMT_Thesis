% Chapter Template

\chapter{Introduction} % Main chapter title

\noindent\textbf{\large Contents:}

\noindent\hrulefill
\noindent\startcontents[chapters]
\noindent\printcontents[chapters]{}{1}{}
\noindent\hrulefill

\label{Chapter1} % Change X to a consecutive number; for referencing this chapter elsewhere, use \ref{ChapterX}

Advances in astronomy come in many forms, from the exploration of new theories to new observational techniques.  But astronomy would not be possible without the main tool of the astronomer, the telescope.  Starting from when Galileo Galilei first pointed his telescope to the night sky, astronomers have been demanding more from their telescopes.  The only way to enhance a telescopes resolving power is to increase the primary aperture of the telescope.  In the early 1900's telescopes were made up of a large single optics for the primary aperture.  However, there is a limit to how large you can make a single optic.  Eventually, these mirrors were becoming so large that the mirrors would sag under their own weight.  Plus these large optics quickly became heavier and needed more mechanical supports.


% \begin{figure}[!h]
% \centering
% \includegraphics[width=12 cm]{directory_here/image_name_here}
% \caption{Oh caption my caption}
% \label{image_label_for_reference_here}
% \end{figure}


\section{Segmented Mirrors}




\section{FPWFS}



\section{Vector Apodizing Phase Plate}






\begin{itemize}
    \item Segmented Mirrors
    \item FPWFS
    \item vAPP
    % \item hcipy
\end{itemize}



%----------------------------------------------------------------------------------------
%	SECTION 1
%----------------------------------------------------------------------------------------
