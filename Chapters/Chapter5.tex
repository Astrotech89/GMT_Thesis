\chapter{Conclusion} % Main chapter title

In conclusion, a successful optical design was engineered within the constraints laid
out by the restrictions of the METIS group and that of adaptive optics systems. 
Despite not getting to a point where a new mechanical design was made, there are rough
dimensions that the optical design can provide.  The optical design has rough dimensions of:

\begin{itemize}
    \item $L \approx 1.2m$
    \item $W \approx 400 mm$
    \item $H \approx 900mm$
\end{itemize}

Of course these dimensions will increase slightly in order to create an
opto-mechanical design stable enough.  The original design provided by Benjamin showed
that a SLAO system could be implemented but showed to be expensive.
One of the driving factors in this research was to take a concept of a SLAO system and
prove that it could be easily manufactured with the least amount of cost.  The design
incorporates all but one, spherical surfaces and is made out of the same glass type.
The glass type chosen was BK7.  BK7 is known as a easily manufacturable material and 
is not brittle.  All components are mounted normal to gravity and should have limited
long-term wear.

The system has shown to have good performance with spot sizes smaller than the Airy
disk as well as wavefront errors less than what was in the requirements.  Given more
time, the wavefront could be further improved.  The WFE, showed that it suffered in
piston.  Simply put, if the lenslet array was translated a few tenths of a micron, the
WFE would be $\pm$ 0.2 waves PTV.

Another issue that needed to be solved was to implement a simple solution for the
change in laser distance.  Similar to the HARMONI AO system, a telescoping fold mirror
system was integrated into the design.  This prevented the necessity of have a moving
camera and glass, making the system less susceptible to failure.

Given more time, there were a few things that could be improved.  The system still
needs a mechanical design to progress further in its development.  However, the
components used in the make up of the original optical design can be used in the new
design.  Both translation stage and cage beams can be used in the mechanical design. 
It was also noted that some lenses may be too thin.  It will be important to go back
and change some lenses to be thicker in order to become more sturdy.

Despite these set backs, the design looks promising with its performance, ease of
manufacturing, and total size.  It is still unclear if the ESO will accept a SLAO
system for METIS.  However, the work done by myself and Benjamin Arcier show that
a SLAO system does have the performance necessary to perform good science and can
be made at a relatively low cost compared to a LTAO system.  