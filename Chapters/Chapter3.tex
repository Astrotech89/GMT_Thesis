% Chapter Template

\chapter{Optical Design} % Main chapter title
\noindent\textbf{\large Contents:}

\noindent\hrulefill
\noindent\startcontents[chapters]
\noindent\printcontents[chapters]{}{1}{}
\noindent\hrulefill
\label{Chapter3} % Change X to a consecutive number; for referencing this chapter elsewhere, use \ref{ChapterX}
% \tableofcontents

% \vspace{5mm}
This section will go over the process to come up with a new more manufactureable, optical design.  


\section{Mirror Systems}
\label{sec:mirror_sys}

The first option explored was to use mirror designs.  The use of mirrors has the 
added benefit of having zero photon loss.  As seen in Figure \ref{fig:reflect_diag}, the 
electric field of the system can be broken down into the incident electric field ($E^i$),
the reflected electric field ($E^r$) and the transmitted electric field ($E^t$).  Using conservation of energy we can get the following equation:
\begin{equation}
    E_0^i + E_0^r - E_0^t = 0
\end{equation}
With mirrors, the transmission component is zero, therefore the transmitted electric field
is equal to the incident electric field ($E_0^i = E_0^r$).  Mirrors also have the advantage of having more mechanical support since it is possible to mount mirrors on the back. 

\begin{figure}[h!]
\centering
\includegraphics[width=8 cm]{Figures/reflection.png}
\caption{A diagram of the electric fields during the interaction with a medium \cite{keller_2019}.}
\label{fig:reflect_diag}
\end{figure}



The first
design looked at was using an Offner relay to zoom the pupil plane.  The design seen
in Figure \ref{fig:full_offner} was the resulting Offner relay.  The initial resulting
image (Figure \ref{fig:full_offner}), initially shows to have tilt in the image.  In ZEMAX
each field can have it's own color associated to it.  While it is not an exact indicator,
the change in color gives an approximation of the image plane.  This is because the
entrance pupil picks up the two fields at the same time, side by side.  So when the color
changes, this is representative of the two fields changing positions.

\begin{figure}[h!]
\centering
\includegraphics[width=8 cm]{Full_offner.png}
\caption{Above is the extent of the Offner Relay design.  The top right optic is the pick off mirror for the light coming from the laser.  From there the image has a 2x zoom by this configuration.  Note that for off-axis mirrors, ZEMAX constructs the entire mirror.}
\label{fig:full_offner}
\end{figure}

The motivation behind the Offner relay was that it was a simple way to zoom the 
pupil image while taking up the least amount of space.  However this soon became 
apparent that this design was not worth pursuing.  While Offner relays offer good
zoom depending on the placement of the mirrors, the change in focal length. In order
to correct for this, there would have to be additional optics to correct for this
since we want the pupil plane at the lenslet array to be as flat as possible.  These
additional optics would more than likely add to the physical dimension of the SLAO
system.  

\begin{figure}[h!]
\centering
\includegraphics[width=14 cm]{offner_tilt.png}
\caption{An image of the first focus after the Offner relay zoom.  We can see that there is
significant tilt in the focal plane where the color change is.} % Explain this more
\label{fig:offner_tilt}
\end{figure}

Other designs were looked into, such as a Gregorian zoom system as well as off axis
parabolas in order to work in the space envelope.  But all showed similar issues
with space and tilt of the planes.  Because of this, going back to a full lens
system was deemed necessary and potentially simpler.  However, this required some 
design changes from the original design.  The original design had
three main lenses of varying size, all with aspherical surfaces.  The goal of this
design was to have an optical layout with all spherical surfaces.  Also due to
the fact that we are only dealing with one wavelength (595nm), the design could be
made with only one glass type.

\section{Optical Constraints}
\label{optical_constraints}

In order to redesign a new system from scratch, it is important to understand what
the system is supposed to do as well as what is constraining the system.  First
thing and maybe most importantly, is to focus the laser spot.  The laser spot is a
known distance from the telescope, therefore it will not be an infinity focus.  We
know that the sodium layer is roughly 90 km above sea-level with a approximate
thickness of 10 km \cite{sodium}.  So for the telescope pointing at Zenith, we can
assume that the laser spot is 90 km above the telescope.  As the telescope changes
zenith angle, the laser tracks the curvature of the Earth's atmosphere.  So the
distance of the laser spot can be computed as:

\begin{equation}
    d_{LGS} = \frac{z_{LGS}}{\cos \zeta}
\end{equation}

Where $d_{LGS}$ is the distance of the laser guide star (LGS), $z_{LGS}$ is the
distance at zenith, and $\zeta$ is the angle away from zenith.  The output of which
can be seen in Figure \ref{fig:laser_dist}.  With this information we can calculate
the difference in focal length between an object at infinity and the laser spot. 
First we take some approximations about the ELT itself:

\begin{enumerate}
    \item Diameter $(D) = 39m$
    \item $f/D = 17.48$
    \item $f = 681.72m$
\end{enumerate}

With this information we can do a first order approximation of where the focal point
of an object will be using a simple formula:

\begin{equation}
    \frac{1}{f} = \frac{1}{object} + \frac{1}{image} \;\;\;\; \rightarrow \;\;\;\; image = \left(\frac{1}{f} - \frac{1}{object}\right)^{-1}
    \label{eq:img_dist}
\end{equation}

\begin{figure}[h!]
\centering
\includegraphics[width=10 cm]{Figures/las_dist.pdf}
\caption{A plot of laser distance versus Zenith angle.  The red lines mark the 45 degree zenith angle equaling a distance of 127km}
\label{fig:laser_dist}
\end{figure}

For an object at infinity, this equation simplifies to $image = f$.  Which in this
case means the image is approximately 682 meters behind the primary mirror.  This is
where the annular mirror mentioned in Chapter \ref{Chapter2} will be located.  Next
is to figure out how far back the laser light will focus.  Since we can assume that
the laser light will come to focus further than the science light, we can subtract
the distance by $682$ meters.  This can be visualized by Figure
\ref{fig:laser_focus}.

\begin{figure}[h!]
\centering
\includegraphics[width=10cm]{Figures/Laser_focus_vs_Laser_dist.pdf}
\caption{A plot of laser focal distance compared to the laser spot distance.  The focal length of the ELT is subtracted to show the distance between the annular mirror and the laser focus.}
\label{fig:laser_focus}
\end{figure}

\subsection{Lenslet Array Calculations}
\label{sec:lenslet_calc}
On the other side, we need to know what we are imaging.  First we note that the
camera chosen for this system is a Large vISble cAmera (LISA) chosen by ESO that has
a 800x800 pixel array with a pixel size of 24 microns.  This gives a total array
size of 19.2mm $\times$ 19.2mm \cite{SAPHIRA}.  The lenslet array will need to have
the following characteristics \cite{arcier}:

\begin{itemize}
    \item $40 \times 40$ subapertures
    \item 20 pixels per subaperture
    \item 10" FoV or 0.5" per pixel
\end{itemize}

The lenslet array needs to image directly onto the detector array to avoid any
aberrations in the pupil plane.  Therefore, the size of the lenslet array will need
to match the size of the detector array of 19.2mm.  The lenslet will subdivide the
pupil plane.  So each lenslet will cover an equivalent of $39m / 40 = 0.975 m$ of
the entrance pupil.  This gives a diffraction limit for each lenslet to be $\theta =
\lambda / d = 0.589\mu m / 0.975m = 0.124"$.  This gives the size of the Airy disk
for each lenslet of 0.124".  The pixel scale required is 0.5"/pixel.  This means
that the pixel size will be $\approx 4$ times the diffraction limit.  This means
that the pixel size needs to be $4 \times 0.589 \mu m = 2.4 \mu m$.  Since the pixel
size needs to be 24 microns, this means that the \textit{f}-number will need to be
equal to 10.  The diameter of each of these lenslets will be $19.2mm / 40 = 0.48mm$.
Since we know what the speed of the lenslet should be, we can calculate the focal
length of each lenslet:

\begin{equation}
    f/D = N \;\;\;\; \rightarrow \;\;\;\; f = N \times D = 10 \times 0.48mm = 4.87mm
\end{equation}

The key characteristics of the lenslet array are as follows:

\begin{itemize}
    \item $f = 4.87mm$
    \item $D = 0.48mm$
    \item $40 \times 40$ array
\end{itemize}


\subsection{Design of First Lens}
\label{sec:L1_des}
Referring back to Chapter \ref{Chapter2} again, we don't want to a system that is
almost 5 meters long just to have the light come to focus.  It is important that the
first optic accelerate the light convergence in order to save space.  For simplicity
sake, we place our first lens one meter away form the annular mirror.  We want the
lens to have an F-number of roughly 3 ($f/D = 3$).  So in order to do this, it was
first necessary to find the diameter of the lens one meter away from the annular
mirror.  The laser spot size is the largest when the telescope is at zenith,
therefore the calculations for this will be done at $\zeta = 0$. I wrote a small
code to determine the size of L1 based on the focal length of the laser and the
speed of a system.  This uses the simple equation $f/D = N$ to determine $D$ based
on a distance behind the focal point.

\begin{lstlisting}
def find_D(D,N,f1,f2):
    dist_new = f1 - f2
    D_new = dist_new / N
    return D_new
    
D_L1 = (find_D(D_ELT,N_ELT,laser_focus[0],dist_to_L1)).to(u.cm)
print(D_L1)

24.04584882426677 cm
\end{lstlisting}

This returned a diameter of 24 cm and from this we can calculate what the focal
length of Lens 1 (L1) should be $f/D = N \; \rightarrow \; f = N \times D = 3 \times 24.05cm
= 72.1 cm$.  With this focal length established, next is to find where the pupil
plane is located and where the focal point is after L1.
\label{op_L1}

\subsection{The Rest of the System}
\label{sec:sys_require}

With the constraints explained in Section \ref{optical_constraints} and Section
\ref{op_L1}, the rest of the system has a very straight forward approach to
calculating the variables of each optic.  First, we calculate the conjugate planes
after L1.  We can do this by using Equation \ref{eq:img_dist} and the output from
Figure \ref{fig:laser_dist}.  

We also want to know where the pupil plane is.  Instead of taking the distance to
the entrance pupil of ELT, we will use the distance from the exit pupil of the ELT.
The exit pupil is defined as "the image of the aperture stop as seen from an axial
point on the image looking through the interposed lenses \cite{Hecht}."  Here the
aperture stop is the primary mirror the ELT itself.  According to ZEMAX, the exit
pupil is located 38.349 meters away from L1.  Since this location does not change
with zenith angle, this distance will stay the same.  The output of these different
planes can be seen in Figure \ref{fig:L1}


\begin{figure}[h!]
\centering
\includegraphics[width=14 cm]{Figures/L1_dim.pdf}
\caption{A plot of the different characteristics of L1.  Note that the pupil plane is constant.}
\label{fig:L1}
\end{figure}

Next (yes you guessed it) comes L2.  L2 needs to focus the pupil plane to infinity.
This is done by locating L2 one focal length away from the pupil plane.  The reason
the pupil plane should be focused to infinity is that the focal plane still varies
in distance with respect to zenith angle.  In the space between L2 and the next
optic, L3, there can be an altitude compensator without impacting the image quality
of the pupil plane.  The focal length of L2 was not so fixed as the rest of the
system.  A few different focal lengths were tested and the output of which can be
seen in Figure \ref{fig:L2}.

In Figure \ref{fig:L2}, there are a range of options for the focal length of L2. 
The choice came down to two factors: 1) The speed of the system, and 2) linearity. 
A focal length of 12.5 cm was deemed a good choice since it gave a f-number of 1.25,
which is slow enough not to add too many aberrations and quick enough to make the
system more compact, and the linearity of the range of focus is a nice thing to
have.  

\begin{figure}[h!]
\centering
\includegraphics[width=14 cm]{Figures/L2_varying_focus.pdf}
\caption{A plot of the back focal distances of the laser spot after L2 with the focal 
length of L2 listed in the legend.}
\label{fig:L2}
\end{figure}

The rest of the system was designed to do two things: 1) pass the light to a beam
size that matches the size of the lenslet array, 2) and then create a telecentric
lens that focuses the pupil image onto the lenslet array.  Telecentricity is an 
optical property that has the pupil image suffer only from defocus and does not 
suffer from demagnification.
%Silly humor, not sure how well it will go over.  May delete
% With these constraints in place and understood, the next step is.  Not quite the case.  To begin withd, a first order must be put in place.  
The next step is to create an optical drawing of the system.  Before looking into
manufacturable optics, we can make an ideal approximation using a paraxial lens.
The paraxial design was first drawn up (Figure \ref{fig:paraxial_draw}) and then
inserted into ZEMAX.

\begin{figure}[h!]
\centering
\includegraphics[width=14 cm]{Figures/paraxial_draw.png}
\caption{A drawing of the paraxial design.  Note that after L2 the focus of the pupil is at infinity.}
\label{fig:paraxial_draw}
\end{figure}

\section{Realistic Design}
\label{sec:reality}

Here I go through the steps necessary to reach a manufacturable optical system.

\subsection{Paraxial Design}
\label{sec:real_paraxial}

A paraxial approximation is another version of the small angle approximation, where
a ray angle $\theta \approx \tan \theta$ \cite{greivenkamp_2004}.  More simply put,
this is the approximation of an ideal lens with no thickness and no radius of
curvature, just a focal length.  With the data from Section
\ref{optical_constraints}, we can make a table of each lenses attributes as seen in
Table \ref{tab:lens_data}.  Next, the data is applied into to get Figure
\ref{fig:paraxial_layout}.


\begin{table}[h!]
\begin{tabular}{|c|c|c|c|c|}
\hline
\textbf{Objecet} & \textbf{$f$} & \textbf{Lens Diameter} & \textbf{Distance to next object} & \textbf{N} \\ \hline
\textbf{L1}      & 72.1 cm      & 24 cm                  & 86 cm                            & 3          \\ \hline
\textbf{L2}      & 12.5 cm      & 10 cm                  & varying                          & 1.25       \\ \hline
\textbf{L3}      & 15 0mm        & 50 mm                   & 100 mm                            & 3          \\ \hline
\textbf{Tele}    & -60 mm        & 19.2mm                 & 0                                & -          \\ \hline
\textbf{LA}      & 4.87 mm       & 19.2 mm                 & 4.87 mm                           & 10         \\ \hline
\end{tabular}
\caption{Table of key dimensions of the SLAO system.  For simplicity sake, the distance between the telecentric lens and the lenslet array is 0 in ZEMAX.}
\label{tab:lens_data}
\end{table}

\begin{figure}[h!]
\centering
\includegraphics[width=14 cm]{Figures/paraxial_layout.png}
\caption{Paraxial layout of the built system.}
\label{fig:paraxial_layout}
\end{figure}

\subsection{Realistic System}
\label{sec:Optical_real}

With the paraxial system in place, next was to introduce optics that had the same
characteristics as the paraxial lens.  ZEMAX has a built in function that will allow
a user to input a diameter and a focal length and will find an off-the-shelf part to
match your specification.  When inserting a manufacturable lens into ZEMAX, there are a few
parameters that make up a basic lens:

\begin{enumerate}
    \item Radius of curvature
    \item Thickness
    \item Material
\end{enumerate}

Zemax has the user build a surface based off of these parameters.  A lens is made up
of two surfaces (Figure \ref{fig:real_lens}).  With the introduction of more
physical lenses, image quality and pupil quality.  The next step is to construct a
proper merit function to constrain the system and let the program optimize itself.


\begin{figure}[h!]
\centering
\begin{subfigure}{.5\textwidth}
  \centering
  \includegraphics[width=6cm]{Figures/real_lens.png}
  \caption{An example of what a lens looks like in ZEMAX}
  \label{fig:real_lens}
\end{subfigure}%
\begin{subfigure}{.5\textwidth}
  \centering
  \includegraphics[width=6cm]{Figures/real_system_1.png}
  \caption{Figure of the same system with real lenses.}
  \label{fig:lens_system_1}
\end{subfigure}
% \caption{}
\label{fig:real_1}
\end{figure}


\subsection{Adding additional lenses}
\label{sec:real_more_lens}

Since the design was not working with substituting single lenses for paraxial ones,
the next idea was to divide a single lens into two lenses by inserting two surfaces
in the middle of a lens.  This would allow ZEMAX to have additional variables at its
disposal to better optimize the system.  An example can be seen in Figure
\ref{fig:2x_lens}.


\begin{figure}[h!]
\centering
\includegraphics[width=12cm]{Figures/2x_L1.png}
\caption{An Example of splitting a lens into two different optical components.}
\label{fig:2x_lens}
\end{figure}

Now that the system has become more complicated, there needs to be more constraints
on the system.  ZEMAX, like most software, will not always do what you want it to at
first.  Unless otherwise specified, ZEMAX has no issue creating lenses inside of
lenses and creating negative space in order to achieve the performance specified. 
As my advisor Remko Stuik said, "there are a million different ways that an
optimization can go wrong, but only one way for it work correctly."  The way to
correctly optimize a system in ZEMAX is to build a merit function. There are a few
things the Merit function was crucial in.  First, the merit function allows the user
to prevent unrealistic designs such as lenses within lenses and negative
thicknesses.  Second, it also allows you to place pupil and focal planes in certain
locations.  This was a crucial aspect of this design since the lenslet array needs
to image the pupil plane.  That means the pupil plane needs to land exactly on the
face of the lenslet array.  Lastly, the spot size.  It is possible to have a system
with excellent pupil quality.  However, if the spot size is not diffraction limited
then the system cannot provide accurate wavefront measurements.


\begin{figure}[h!]
\centering
\includegraphics[width=14cm]{Figures/wo_merit_function.png}
\caption{An Example of how a system can "blow up" without the correct Merit Function.}
\label{fig:wo_merit}
\end{figure}

With a proper Merit function, designing a proper system was just an iterative
process of optimizing, patience, understanding what had changed, and what function
was driving those changes.  After optimization, the system was not providing an
acceptable wavefront in the pupil plane.  The next step was to find where an aspherical
surface could be located in order to provide the desired performance.  This goes in
contrast of the original goal of this thesis.  Originally, the system was to have no
aspherical lenses.  However, having one aspherical surface could prove to be cheaper than
adding additional optics that require space, and mounting.  ZEMAX has a feature called
"Find Best Asphere", where it goes through each optical surface and attempts to drive the
Merit Function as low as possible.  While running through the whole system would be a
definitive way to determine the best surface, a close look at the system can show where the
best surface would be (Figure \ref{fig:good_pupil}).


\begin{figure}[h!]
\centering
\includegraphics[width=14cm]{Figures/good_pupil.png}
\caption{An image of the last few optics in the system.  Where the color changes from green to blue is a rough indicator of where the pupil plane is.}
\label{fig:good_pupil}
\end{figure}

In Figure \ref{fig:good_pupil}, we can see where the pupil plane is.  Here there is
still a paraxial representation of the lenslet array.  At this location there is
still high wavefront error.  The surface just before the lenslet array would be the
ideal candidate.  If the asphere could match the wavefront error, then there would
be a flat wavefront reaching the lenslet array.  To be safe, ZEMAX ran through each
surface in Figure \ref{fig:good_pupil} to determine which was the best asphere.  The
back of the telecentric lens was indeed the best candidate.  From that the inline
design optimized to reduce the Peak-to-Valley (PTV) wavefront and drive the spot
size down to the diffraction limit.

The design shown in Figure \ref{fig:full_lens} showed that a design could be made to
accurately do wavefront sensing with only one aspherical surface.  The next step was
to implement a simpler way to extend the light path for the changing laser distance.

\begin{figure}[h!]
\centering
\includegraphics[width=14cm]{Figures/full_lens.png}
\caption{A layout of the in line WFS.}
\label{fig:full_lens}
\end{figure}


First, it is important to know where the light path can be extended without effect
system performance.  The pupil image quality is what needs to be preserved.  Going
back to Figure \ref{fig:paraxial_draw}, we can see that the pupil image is focused
to infinity between L2 and L3.  This is where the system can correct for the
changing laser distance.  When looking at concepts to compensate for the changing
altitude, we look to other instruments going onto the ELT.  HARNOMI impliments a
telescoping mirror system to extend the light path as seen in Figure
\ref{fig:harmoni}.

\begin{figure}[h!]
\centering
\includegraphics[width=14cm]{Figures/HARMONI_AO.jpg}
\caption{A layout of the HARMONI WFS.  In here the use a telescoping mirror system labeled as Altitude Focus Unit \cite{harmoni}.}
\label{fig:harmoni}
\end{figure}

A similar system was implemented in the design of the METIS SLAO system.  This meant
extending the distance between L2 and L3.  However, with a stable merit function,
this was easily implemented (Figure \ref{fig:SLAO_comp}).

\begin{figure}[h!]
\centering
\includegraphics[width=14cm]{Figures/shaded_zoom.png}
\caption{A layout of the METIS SLAO altitude compensator.}
\label{fig:SLAO_comp}
\end{figure}

Next, the system needed to be more compact.  One of the requests of the METIS SLAO
system was to make the structure as compact as possible.  The simplest way to
achieve this was to add $45^{o}$ fold mirrors in between the lenses.  Flat, $45^{o}$
mirrors only create coordinate changes, but do not effect the image quality.  By
adding the fold mirrors, the almost 3 meter long system was reduced to less than a
meter long.  The full system with fold mirrors and altitude compensator can be seen
in Figure \ref{fig:SLAO_trace}.

\begin{figure}[h!]
\centering
\includegraphics[width=14cm]{Figures/system_trace.png}
\caption{A layout of the full METIS SLAO design.}
\label{fig:SLAO_trace}
\end{figure}

The last step was to replace the paraxial lenslet array.  With the spot size and
wavefront within tolerance, it was time to substitute in a realistic lenslet array.
Adding a lenslet array into ZEMAX can be done multiple ways.  Normally ZEMAX allows
for a non-sequential component to be added into a sequential system.  The way to do this
in ZEMAX is to select "Non-sequential Component" in the "Surface Type" section.  From this
you can open up the "Non-sequential Component Editor".  In this you can normally add as
many non-sequential components as you'd like.  From there, an exit port surface needs to be
made in order for ZEMAX to go back to calculating sequentially\cite{zemax_non_seq}. 
Despite making practice files and making them work outside of the system, it continued to
fail when attempting to add it to the layout.

The next way was to use a "User Defined" surface.  This option allows for the
creation of a lenslet array in a sequential manner.  However, this method does not
allow for stable optimization in ZEMAX.  The output from a lenslet array is of
course, an array of spots.  ZEMAX wants to have all of the rays through a pupil
converge into one spot.  Therefore, optimization was not an option.  Using the
calculations from \ref{sec:lenslet_calc}, a lenslet array was designed from these
parameters.  From there, I used the slider function in ZEMAX to nudge the image
plane after the lenslet array into focus.  With the design in place, we can begin to
look at the kind of performance it will achieve.
